\documentclass{article}
\usepackage[left=1in,right=1in,bottom=1in,top=1in]{geometry}
\usepackage{amsmath}

\begin{document}

\subsubsection*{Response to comments by Reviewer 1}

Thank you very much for your detailed reading of the paper and your insightful comments and questions. You raise a number of good points and we have modified the manuscript in several places based upon your feedback.  In our opinion, these revisions have nicely improved upon the original submission.

\begin{enumerate}

\item \emph{Numerical studies only focus on the computational performances of the algorithms. One may be interested in some details in the proposed algorithm: the size of variables remaining from the safe or strong rules, the number of KKT condition checks, the adaptive batch sizes, etc. It would be good to add the numerical studies for details in the algorithm.}

\item \emph{The paper assumes that the predictive variables are standardized for linear and logistic regression models. For group lasso, the additional orthogonal assumption are required for the proposed method. I’m just wondering if the proposed method is applicable without the standardization assumptions. Specifically, is it possible to develop (or exist) the sequential safe and strong rules without the standardization assumptions.}

\item \emph{In page 10, line 13, ``We also assume that post-convergence check will usually pass in the first iterations''. I’m not sure it is reasonable.}

\item \emph{In page 8, line 19, ``the reference solutions are continually updated''. The reference solutions are not always updated. So, I’m not sure the term ``continually'' is appropriate.}

\item \emph{In Figure 3, the green line (HSSR) pattern is quite different with others. Could you explain it?}

\item \emph{In Section 6(Conclusion), there does not exist the sequential safe rules for sparse Cox regression, Poisson regression, and support vector machines. It would be good to explain the existence of the sequential strong rules for each problem.}

\item \emph{In Theorem 4.1, $x_* \in \{ x_j : \hat{\beta}_j(\lambda_l) \ne 0 \}$. Is the $x_*$ arbitrary? It seems that further explanation for $x_*$ is needed.}

\item \emph{Minor corrections:} These have all been made. Thank you very much for pointing them out.

\end{enumerate}

\end{document}



